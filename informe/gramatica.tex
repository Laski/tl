El esquelto para hacer la gramática\footnote{implementada en parser.y} fue tomado de la página citada en la presentación\footnote{http://gnuu.org/2009/09/18/writing-your-own-toy-compiler/}. Sobre este esqueleto armamos nuestra gramática, que tiene varios puntos de diferencia con la que muestran de ejemplo en ese sitio.

Nuestra gramática parte de un símbolo incial $programa$ que a su vez se divide en una lista de $funciones$ y una instrucción de $ploteo$.

Una función consiste en la palabra 'function' (representada por el token terminal TFUNC), un $nombre$ (que no es más que un terminal que matchea con strings no empezados por números), una lista de $argumentos$ (lista de $nombre$s separadas por comas) entre paréntesis y un $bloque$ de código.

Un $bloque$ es una lista (quizás vacía) de sentencias encerradas entre llaves o una única sentencia.

Una $sentencia$ puede ser una $asignacion$ (un $nombre$, un símbolo de $=$ y una $expresión$), una sentencia $ifthenelse$, un $while$ (que a su vez tiene una $condicion$ y otro $bloque$) o un $return$ (que lleva a su lado una $expresion$).

El caso del $ifthenelse$ requiere un análisis específico. Un $ifthenelse$ puede tener o no tener 'else', lo cual nos trae una ambiguedad a la hora de parsear cadenas del estilo (intencionalmente sin indentar):
\begin{verbatim}
if cond1 then
if cond2 then
codigo1
else
codigo2
\end{verbatim}
que se refleja en el parser en un conflicto shift/reduce, como ya explicamos anteriormente (junto con la forma de resolución que elegimos). Es decir: nuestra gramática es ambigua.

Una $expresion$ puede ser un $numero$ (double o $\pi$), una $llamada_funcion$ (el nombre de una función con más $expresiones$ entre paréntesis como argumentos) u operaciones aritméticas entre expresiones. Para salvar las ambiguedades propias de esta parte de la gramática le indicamos a BISON la precedencia de los operadores\footnote{basándonos en el código de ejemplo de http://www-h.eng.cam.ac.uk/help/tpl/languages/flexbison/}.

Una $condicion$ puede ser una comparación entre dos $expresion$es o varias condiciones unidas con operadores lógicos.

Por último, la instrucción $ploteo$ consiste en dos $llamada\_funcion$ con un $nombre$ de variable y tres $expresion$es que reflejan el inicio, el paso y el final.

La gramática se complementa con un archivo\footnote{tokens.l} con directivas para el tokenizador, que le permite convertir símbolos terminales en los tokens que usamos, ignora saltos de línea y maneja números enteros, floats y cadenas arbitrarias (como los nombres de variables).

Entre los problemas que encontramos hubo uno que nos presento dificultades: lograr armar la expresión regular para que el tokenizador ignore los comentarios multilínea. Finalmente gracias a los comentarios del corrector pudimos dar con la expresión regular acertada.