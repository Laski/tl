El objetivo del Trabajo Práctico es crear un lenguaje de programación imperativo que permita definir funciones que serán evaluadas y nos permitirán traficar curvas paramédicas en el plano. 

Para esto, tuvimos que decidir e implementar los siguientes puntos:
\begin{itemize}
\item La gramática que vamos a usar para la implementación
\item El analizador léxico que recibe el código fuente
\item El parser que toma los tokens y verifica que coincida con la gramática
\item El Abstract Syntax Tree que permite representar las funciones del código fuente
\end{itemize}
Para la implementación decidimos utilizar BISON y C++ y utilizamos como referencia la documentación de BISON y el blog que fue recomendado.
