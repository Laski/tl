Como primer conclusión general podemos afirmar que el lenguaje propuesto por la cátedra es LR1, ya que pudimos construir un parser LR1 que lo reconoce. Además podemos decir que, a pesar de su simpleza, es un lenguaje potente que no solo permite el uso de las clásicas estructuras de control de flujo sino que también, como hemos visto, permite definir funciones de manera recursiva.

Por otro lado el desarrollo de la implementación de este lenguaje de programación nos permitió terminar de comprender las diferentes etapas que son inherentes a cualquier compilador o intérprete, desde el reconocimiento de la cadena de entrada y la generación del arbol sintáctico, hasta dar semántica a este. 

En ese sentido un punto interesante a remarcar es la relativa independencia de estas etapas. Por ejemplo, si en lugar de interpretar el codigo, simularlo y generar una salida apta para graficar quisieramos traducir las funciones a codigo máquina para generar un binario ejecutable sólo deberíamos modificar, en principio, el código que se encarga de manejar el árbol sintáctico una vez construido.

Por último, también descubrimos y pudimos valorar la inmensa utilidad de tener otros lenguajes de programación (en nuestro caso C++) para poder parsear nuevos. Nos compadecemos de quien haya tenido que escribir el primer compilador.