Como primer conclusi�n general podemos afirmar que el lenguaje propuesto por la c�tedra es LR1, ya que pudimos construir un parser LR1 que lo reconoce. Adem�s podemos decir que, a pesar de su simpleza, es un lenguaje potente que no solo permite el uso de las cl�sicas estructuras de control de flujo sino que tambi�n, como hemos visto, permite definir funciones de manera recursiva.
Por otro lado el desarrollo de la implementaci�n de este lenguaje de programaci�n nos permiti� terminar de comprender las diferentes etapas que son inherentes a cualquier compilador o int�rprete, desde el reconocimiento de la cadena de entrada y la generaci�n del arbol sint�ctico, hasta etapa de dar sem�ntica a este. 

En ese sentido un punto interesante a remarcar es la relativa independencia de estas etapas. Por ejemplo, si en lugar de interpretar el codigo, simularlo y generar una salida apta para graficar quisieramos traducir las funciones a codigo m�quina para generar un binario ejecutable s�lo deber�amos modificar, en principio, el c�digo que se encarga de manejar el �rbol sint�ctico una vez construido.